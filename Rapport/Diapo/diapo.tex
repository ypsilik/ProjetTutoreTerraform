\documentclass[11pt]{beamer} %modèle du document avec comme option la taille de la police
% Liste des Packages :

\usepackage[utf8]{inputenc} % encodage de caractères
\usepackage[OT1]{fontenc} % encodage de caractères
\usepackage[french]{babel} % support du français
\usepackage{textcomp}
\usepackage{color}
\usepackage{listings}
\usepackage{caption}
\usepackage{fancyhdr}

\usetheme{Hannover} %PaloAlto  thème principal
\setbeamercolor{background canvas}{bg=yellow!10!white} % couleur de fond

\title[Projet Tutoré]{Déploiement et provisionnement d’un IAAS avec Openstack et Terraform}
\author{Valentin \textsc{Norberto da silva}, Maud \textsc{Laurent} et Valentin \textsc{Peyregne}}
\institute{IUT Nancy Charlemagne}
\date{\today}
\addtobeamertemplate{footline}{\hfill\insertframenumber/\inserttotalframenumber} % numerotation diapo

\begin{document}
	
	% Page de garde :
		\begin{frame}
		\titlepage
			\begin{flushright}
				\includegraphics[scale=0.3]{logo.png}
			\end{flushright}			
			%\thispagestyle{empty} % pas numerotation first page
		\end{frame}

	% Sommaire :

		\begin{frame}
			\transdissolve[duration=2]
			\frametitle{Sommaire}
			\tableofcontents%[currentsection, hideothersubsections, pausesubsections]		
		\end{frame}


	\section{Introduction et problématique}
		
		% Diapo 2 : Contexte
		
		\subsection{Contexte}
			\begin{frame}
				\frametitle{La société Xilopix}
					
				\begin{figure}
					\begin{center}
						\includegraphics[scale=0.8]{img2.png}
					\end{center}
				\end{figure}
				
				\begin{itemize}
					\item Start-up qui développe un moteur de recherche
					\item Combinaison d'éléments de différentes natures lors d'une recherche
					\item Amélioration de la pertinence des résultats
				\end{itemize}
			\end{frame}
		
		% Diapo 3 : Problématique
		
			\begin{frame}
				\frametitle{Problématique \& motivations}
				\begin{itemize}
					%\item Xilopix utilise une infrastructure cloud avec plusieurs providers tous sur OpenStack
					\item Créer une \textit{"infrastructure as code"}
					\item Réduire les dépendances avec les hébergeurs
					%Souhaite ne Souhaite plus être dépendant d'OpenStack
					\item Automatiser la création de client OpenStack
					%Souhaite pouvoir créer plusieurs instances d'Openstack 
					%\item Utilisation de Terraform
				\end{itemize}
			\end{frame}
			
		\subsection{Terraform}
			\begin{frame}{Terraform}
			\begin{itemize}
			\item Centralise l'architecture d'une infrastructure
			\item Création, gestion d'infrastructures
			\item Création de recettes
			\end{itemize}
			\begin{figure}
				\begin{center}
					\includegraphics[scale=0.3]{logoTerraform.jpg}
				\end{center}
			\end{figure}
			\end{frame}
		
		% Diapo 5 : Objectifs du projet 
		
		\subsection{Les objectifs du projet}
			\begin{frame}
				\frametitle{Les objectifs}
				\begin{itemize}
					\item Développer un proof of concept sur l'outil Terraform
					\item Créer cluster 4 VM clientes OpenStack grâce à Terraform
					%Mettre en place une infrastructure semblable à celle de Xilopix, avec un provider (cloudwatt) et un cluster de quatre machines virtuelles clientes Openstack avec Terraform
				\end{itemize}
			\end{frame}
		
	\section{L'avancée du Projet}
	
	\subsection{Comparaison}
			% Diapo 8 : Comparaison entre les tâches réalisées / en cours et ce qui était attendu
		
		\begin{frame}{Comparaison}

		\begin{columns}[c]
            \begin{column}{5cm}
         
          {\small \textbf{ Objectifs attendus}}
                \begin{itemize}
                \item {\small Compréhension des outils utilisés}                                                    
                \item {\small Proof of concept Terraform}
               \end{itemize}
            \end{column}
            \begin{column}{5cm}
          {\small   \textbf{ Tâches réalisées}}
            \begin{itemize}
            
            \item  {\small Documentation sur les outils }
            \item  {\small Installation et création d'une machine avec python-nova}
            \item  {\small Création et application de recette de base Terraform }
            \item  {\small  Gestion Ip flottantes }
            \item  {\small Documentation Terraform (technique) }
            \end{itemize}
               
                \end{column}
          \end{columns}
\end{frame}

	
		% Diapo 6 : Tâches réalisées
		
		\subsection{Tâches réalisées}
			\begin{frame}
				\frametitle{Les tâches réalisées}
				\begin{itemize}
					\item Création d'un dépôt Git
					\item Documentation sur python-nova, OpenStack et Terraform
					\item Installation du client python-nova
					\item Création et application de recette de base Terraform
				\end{itemize}
			\end{frame}
		
		% Diapo 7 : Tâches en cours 
		
		\subsection{Tâches en cours}
			\begin{frame}
				\frametitle{Les tâches en cours}
				\begin{itemize}
					\item Gestion des IP flottantes
					\item Création de règles pour le security group
					\item Documentation Terraform (plus technique)
				\end{itemize}
			\end{frame}
		
			
		% Diapo 9 : Tâches prévues 
		
		\subsection{Tâches à prévoir}
			\begin{frame}
				\frametitle{Ce qu'il reste à faire}
				\begin{itemize}
					\item Création de réseau et sous-réseau
					\item Création d'un routeur pour accéder à internet
					\item Provisionnement avec Ansible
				\end{itemize}
			\end{frame}
			
			
			\begin{frame}
			\begin{center}
			\Huge{Questions ?}
			\end{center}
			\end{frame}
\end{document}
