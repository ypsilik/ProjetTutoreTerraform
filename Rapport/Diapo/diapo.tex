\documentclass[11pt]{beamer} %modèle du document avec comme option la taille de la police
% Liste des Packages :

\usepackage[utf8]{inputenc} % encodage de caractères
\usepackage[OT1]{fontenc} % encodage de caractères
\usepackage[french]{babel} % support du français
\usepackage{textcomp}
\usepackage{color}
\usepackage{listings}
\usepackage{caption}

\usetheme{PaloAlto} % thème principal
\setbeamercolor{background canvas}{bg=yellow!10!white} % couleur de fond

\title[Projet Tutoré]{Déploiement et provisionnement d’un IAAS avec Openstack et Terraform}
\author{Valentin \textsc{Norberto da silva}, Maud \textsc{Laurent} et Valentin \textsc{Peyregne}}
\institute{IUT Nancy Charlemagne}
\date{\today}

\begin{document}
	
	% Page de garde :
		\begin{frame}
			\titlepage
			\begin{flushright}
				\includegraphics[scale=0.3]{logo.png}
			\end{flushright}
		\end{frame}

	% Sommaire :

		\begin{frame}
			\transdissolve[duration=2]
			\frametitle{Sommaire}
			\tableofcontents%[currentsection, hideothersubsections, pausesubsections]
		\end{frame}


	\section{Introduction et problématique}
		
		% Diapo 1 : le cloud computing
		
		\subsection{Le cloud computing}
			\begin{frame}
				\frametitle{Le cloud computing}
				\begin{itemize}
					\item Permet d'utiliser des ressources informatiques sans les posséder ou encore d'accéder à des services
					\item On distingue 3 catégories de services :\\
						- le cloud privé \\
						- le cloud public \\
						- le cloud hybride \\
					\item IaaS / PaaS et SaaS
				\end{itemize}
			\end{frame}

		% Diapo 2 : Contexte
		
		\subsection{Contexte}
			\begin{frame}
				\frametitle{La société Xilopix}
					
				\begin{figure}
					\begin{center}
						\includegraphics[scale=0.8]{img2.png}
					\end{center}
				\end{figure}
				
				\begin{itemize}
					\item Start-up qui développe un moteur de recherche
					\item Combinaison d'éléments de différentes natures lors d'une recherche
					\item Amélioration de la pertinence des résultats
				\end{itemize}
			\end{frame}
		
		% Diapo 3 : Problématique
		
			\begin{frame}
				\frametitle{Problématique \& motivations}
				\begin{itemize}
					\item Xilopix utilise une infrastructure cloud avec plusieurs providers tous sur OpenStack
					\item Souhait de ne plus être dépendant d'OpenStack 
					\item Utilisation de Terraform
				\end{itemize}
			\end{frame}
		
		% Diapo 5 : Objectifs du projet 
		
		\subsection{Les objectifs du projet}
			\begin{frame}
				\frametitle{Les objectifs}
				\begin{itemize}
					\item Développer un proof of concept sur l'outil Terraform
					\item Mettre en place une infrastucture semblable à celle de Xilopix, avec un provider (cloudwatt) et un cluster de quatres machines virtuelles cliente Openstack avec Terraform
				\end{itemize}
			\end{frame}
		
	\section{L'avancée du Projet}
	
	\subsection{Comparaison}
			% Diapo 8 : Comparaison entre les tâches réalisées / en cours et ce qui était attendu
		
		\begin{frame}{Comparaison}
		Objectif attendu 
			\begin{itemize}
				\item Compréhension des outils utilisés
				\item Proof of concept fonctionnel de 3 machines créées avec Terraform et une avec python-nova
			\end{itemize}
		Tâches réalisées
				\begin{itemize}
					\item \textbf{Documentation sur python-nova, OpenStack et Terraform}
					\item \textbf{Installation du client python-nova}
					\item \textbf{Création d'une machine avec python-nova}
					\item \textbf{Création et application recette de base Terraform}
					\item Gestion des IPs flotantes
					\item Documentation Terraform (plus technique)
				\end{itemize}
		\end{frame}
			
		
		% Diapo 6 : Tâches réalisées
		
		\subsection{Les tâches réalisées}
			\begin{frame}
				\frametitle{Les tâches réalisées}
				\begin{itemize}
					\item Création d'un dépôt Git
					\item Documentation sur python-nova, OpenStack et Terraform
					\item Installation du client python-nova
					\item Création et application recette de base Terraform
				\end{itemize}
			\end{frame}
		
		% Diapo 7 : Tâches en cours 
		
		\subsection{Les tâches en cours}
			\begin{frame}
				\frametitle{Les tâches en cours}
				\begin{itemize}
					\item Gestion des IPs flotantes
					\item Documentation Terraform (plus technique)
				\end{itemize}
			\end{frame}
		
			
		% Diapo 9 : Tâches prévues 
		
		\subsection{Les tâches à prévoir}
			\begin{frame}
				\frametitle{Ce qu'il reste à faire}
				\begin{itemize}
					\item Création de réseau et sous-réseau
					\item Création d'un routeur pour accéder à internet
					\item Provisionnement avec Ansible
				\end{itemize}
			\end{frame}
		



\end{document}
